\chapter{Fragen}
Müssen Alarme umgesetzt werden? ->Müssen nicht berücksichtigt werden
Müssen Hardwareconfigs von der CPU oder von TMFast gehandhabt werden? ->hardwareconfig der (z.B. Ventile) ist in VHDL (andere Kommunikationsprotokole und co pro z.B. Ventile)
Wie verhalten sich die In- und Outputs? 
\begin{itemize}
    \item Sollen alle Eingänge Zähl- und Richtungseingang sein oder Aufteilung zwischen den beiden? Nur zähler
    \item 
\end{itemize}

wie funktionieren ventile überhaupt?

bit an die cpu senden 

nachlauf bei inkrement zum abfüllen

cpu macht  - tm fast macht ventil zu

auf ist 1 zu ist low für ventil

abfüllende menge muss übergeben werden

vhdl auf hoher ebene darstellen (als baustein)

syncron oder asyncron (für zähler oder generell) -> Alles Syncron (TM Fast arbeitet (einstellbar mit 50mHz) schneller als ein ventil) -> dadurch zählt es schneller als ikremente des ventils

clk wird bei änderung aktiviert (High->Low / Low->High)

steuerung des ventils über PLC oder TMFAST? Erstamal Plc mit bool senden (überlauf für counter (justierung)) / TMFast erfordert Kommunikations-Protokolle

Filter?

Anwendung in mehreren oder einem Process in vhdl?

soll der timer nur zählen, wenn die flanke positiv wird? oder auch wenn sie negativ wird?

\chapter{Tests}

rückgabe über das (fb\_if) feedback\_interface

EncEmu encoder für impulse 

schwierigkeit: di1 oder di0 zählen

RD\_Rec sind azyklisch und nur selten benutzt

library test 8 counter -> name mit test cases Jürgen, ..., Kollege aus Prag

Muss ich nicht allein schreiben

Mathias fragen wenn funktionierendes Programm -> in den Nightly test muss in den nightly build

-systemlogic updaten auf v2 

-zähler des sea teams verwenden

- 0000\_0000\_0000\_0000\_0000\_0000\_0000\_0001 -> FB\_IF(0)(0)<='1';
-    /->Q0.0                          (FB\_IF(0))

- FB\_IF(1)(0) == QD4
\chapter{Passwörter}

PLC: Siemens123