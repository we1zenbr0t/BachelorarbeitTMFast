\chapter{Plan} 

Hier sind einige Schritte, die Ihnen dabei helfen können, eine VHDL-Applikation für das TM FAST-Modul zu entwickeln, um das alte 
FM350-2-Modul zu ersetzen:

\subsection*{Anforderungsanalyse} 
\begin{itemize} 
    \item Analysieren Sie sorgfältig die Funktionalität und Schnittstellen des FM350-2-Moduls. 
    \item Identifizieren Sie die Kernfunktionen, die das neue TM FAST-Modul erfüllen muss. 
    \item Definieren Sie klar die Eingangs- und Ausgangssignale, die Kommunikationsprotokolle und andere relevante Spezifikationen. 
\end{itemize}

\subsection*{Architekturentwurf} 
\begin{itemize} 
    \item Entwerfen Sie eine modulare Architektur für Ihre VHDL-Applikation. 
    \item Unterteilen Sie die Funktionalität in logische Blöcke wie Steuerungslogik, Kommunikationsschnittstellen, Datenpfade usw. 
    \item Legen Sie die Schnittstellen zwischen den einzelnen Modulen fest. 
\end{itemize}

\subsection*{VHDL-Entwicklung} 
\begin{itemize} 
    \item Beginnen Sie mit der Implementierung der einzelnen VHDL-Module basierend auf Ihrem Architekturentwurf. 
    \item Achten Sie auf eine strukturierte und modulare Codeerstellung, um die Wartbarkeit und Erweiterbarkeit zu erleichtern. 
    \item Verwenden Sie VHDL-Sprachkonstrukte wie Prozesse, Signale, Typen und Pakete, um Ihre Applikation übersichtlich und 
    effizient zu gestalten. 
\end{itemize}

\subsection*{Simulation und Test} 
\begin{itemize} 
    \item Erstellen Sie umfangreiche Testbänke, um Ihre VHDL-Applikation gründlich zu überprüfen. 
    \item Führen Sie Simulationen durch, um das korrekte Verhalten der einzelnen Module und der Gesamtapplikation zu verifizieren. 
    \item Testen Sie die Applikation schrittweise, um mögliche Fehler frühzeitig zu erkennen und zu beheben. 
\end{itemize}

\subsection*{Integration und Validierung} 
\begin{itemize} 
    \item Integrieren Sie Ihre VHDL-Applikation in die Gesamtarchitektur des TM FAST-Moduls. 
    \item Testen Sie die Applikation in der realen Umgebung, um ihre Kompatibilität und Funktionalität zu validieren. 
    \item Passen Sie die Applikation gegebenenfalls an, um Anforderungen und Spezifikationen des TM FAST-Moduls zu erfüllen. 
\end{itemize}

\subsection*{Dokumentation und Versionskontrolle} 
\begin{itemize} 
    \item Erstellen Sie eine ausführliche Dokumentation, die die Architektur, Schnittstellen, Konfiguration und Verwendung Ihrer 
    VHDL-Applikation beschreibt. 
    \item Nutzen Sie ein geeignetes Versionskontrollsystem, um Änderungen an Ihrer Applikation nachzuverfolgen und zu verwalten. 
\end{itemize}

Durch diese strukturierte Vorgehensweise können Sie sicherstellen, dass Ihre VHDL-Applikation für das TM FAST-Modul die 
Anforderungen des FM350-2-Moduls erfüllt und in die Gesamtarchitektur des TM FAST-Systems nahtlos integriert wird.

\chapter{Fragen}
Müssen Alarme umgesetzt werden? ->Müssen nicht berücksichtigt werden
Müssen Hardwareconfigs von der CPU oder von TMFast gehandhabt werden? ->hardwareconfig der (z.B. Ventile) ist in VHDL (andere Kommunikationsprotokole und co pro z.B. Ventile)
Wie verhalten sich die In- und Outputs? 
\begin{itemize}
    \item Sollen alle Eingänge Zähl- und Richtungseingang sein oder Aufteilung zwischen den beiden? Nur zähler
    \item 
\end{itemize}

wie funktionieren ventile überhaupt?

bit an die cpu senden 

nachlauf bei inkrement zum abfüllen

cpu macht  - tm fast macht ventil zu

auf ist 1 zu ist low für ventil

abfüllende menge muss übergeben werden

vhdl auf hoher ebene darstellen (als baustein)

syncron oder asyncron (für zähler oder generell) -> Alles Syncron (TM Fast arbeitet (einstellbar mit 50mHz) schneller als ein ventil) -> dadurch zählt es schneller als ikremente des ventils

clk wird bei änderung aktiviert (High->Low / Low->High)

steuerung des ventils über PLC oder TMFAST? Erstamal Plc mit bool senden (überlauf für counter (justierung)) / TMFast erfordert Kommunikations-Protokolle

Filter?

Anwendung in mehreren oder einem Process in vhdl?

soll der timer nur zählen, wenn die flanke positiv wird? oder auch wenn sie negativ wird? -> nur bei positiv

Allgemein:
Schriftgröße und Seitenränder?
Zitieren von Handbüchern (Graue Literatur) und ChatGPT?

\chapter{Tests}

rückgabe über das (fb\_if) feedback\_interface

EncEmu encoder für impulse 

schwierigkeit: di1 oder di0 zählen

RD\_Rec sind azyklisch und nur selten benutzt

library test 8 counter -> name mit test cases Jürgen, ..., Kollege aus Prag

Muss ich nicht allein schreiben

Mathias fragen wenn funktionierendes Programm -> in den Nightly test muss in den nightly build

-systemlogic updaten auf v2 

-zähler des sea teams verwenden

- 0000\_0000\_0000\_0000\_0000\_0000\_0000\_0001 -> FB\_IF(0)(0)<='1';
-    /->Q0.0                          (FB\_IF(0))

- FB\_IF(1)(0) == QD4
\chapter{Passwörter}

PLC: Siemens123

\href{https://siemens.sharepoint.com/teams/P0000045/Backbone/Abschlussarbeiten/Masterarbeit_Andreas_Kick.pdf}{Masterarbeit von Andreas Kick}
\href{https://siemens.sharepoint.com/teams/P0000045/Backbone/Abschlussarbeiten/Bachelorarbeit_Thomas_Weisel.pdf}{Bachelorarbeit von Thomas Weisel}