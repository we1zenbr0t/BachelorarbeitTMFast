\chapter{Einleitung} 

\section{Problemstellung und Motivation} 
Die Produktserie S7-300, insbesondere das FM350-2 Modul, sind nicht mehr in der aktiven Vermarktung und dadurch nicht mehr bestellbar. Das FM350-2 Modul bietet 8 Kanäle für 
hochkanalige Dosieranwendungen und ist nur noch als Ersatzteil verfügbar. Ein alternatives Lösungskonzept ist der Einsatz des TM FAST Moduls, eines freiprogrammierbaren 
Hochgeschwindigkeits-Moduls, welches eine erweiterte Kanalanzahl von 10 bis 12 Kanälen ermöglicht. Ziel dieser Bachelorarbeit ist die Entwicklung eines VHDL-basierten 
Applikationsbeispiels zur Demonstration der Einsatzmöglichkeiten des TM FAST Moduls in hochkanaligen Dosieranwendungen. Gleichzeitig soll die Integration in das TIA Portal 
erfolgen, um eine reibungslose Nutzung in industriellen Automatisierungsumgebungen zu gewährleisten. 

Als dualer Student bei Siemens verfüge ich bereits über Erfahrung in der industriellen Automatisierungstechnik. Diese Arbeit bietet mir die Gelegenheit, mein Wissen in den 
Bereichen Embedded Systems, Industrieautomation und Software Engineering weiter zu vertiefen. Ich freue mich darauf, dieses praxisnahe Thema in Kooperation mit Siemens zu 
bearbeiten und einen wertvollen Beitrag zur Weiterentwicklung industrieller Automatisierungslösungen zu leisten. 
%//////////////////////////////////////////////////////////////////////////////////////////////////////////////////////
%//////////////////////////////////////////////////////////////////////////////////////////////////////////////////////
\section{Zielsetzung der Arbeit} 
Ziel dieser Bachelorarbeit ist die Entwicklung eines VHDL-basierten Applikationsbeispiels zur Ablösung des FM350-2 Moduls durch das TM FAST Modul. Dabei soll die Kanalanzahl von 8 
auf 10 oder 12 erhöht werden, um eine flexiblere Nutzung in hochkanaligen Dosieranwendungen zu ermöglichen. Ein wesentlicher Bestandteil der Arbeit ist zudem die Integration des 
entwickelten Applikationsbeispiels in das TIA Portal, um eine nahtlose Einbindung in bestehende Automatisierungslösungen zu gewährleisten. Neben der Implementierung des VHDL-Codes 
werden bewährte Methoden des Software Engineerings angewendet, um die Code-Qualität und Skalierbarkeit sicherzustellen. Schließlich soll eine umfassende Dokumentation erstellt 
werden, die eine detaillierte Beschreibung der Nutzung und Inbetriebnahme des Moduls umfasst. 
%//////////////////////////////////////////////////////////////////////////////////////////////////////////////////////
%//////////////////////////////////////////////////////////////////////////////////////////////////////////////////////
\section{Methodisches Vorgehen und Struktur der Arbeit} 
Zur Umsetzung der Ziele werden zunächst die bestehenden Funktionalitäten des FM350-2 Moduls analysiert und an den aktuellen Erfordernissen der Kunden gespiegelt. Dabei werden sowohl 
die Hardware- als auch die Software-Architektur untersucht und relevante Funktionen für die Dosieranwendung identifiziert. Anschließend erfolgt die Entwicklung des VHDL-basierten 
Applikationsbeispiels, wobei die ausgewählten Funktionen des FM350-2 Moduls in VHDL umgesetzt und die Kanalanzahl auf 10 oder 12 erweitert wird. Danach wird die Integration in das 
TIA Portal vorgenommen, indem das TM FAST Modul an das Automatisierungssystem angebunden und geeignete Schnittstellen sowie Datenstrukturen entwickelt werden. 
Qualitätssicherungsmaßnahmen spielen ebenfalls eine zentrale Rolle. Hierbei werden bewährte Entwicklungswerkzeuge wie Intel Quartus Prime und das TIA Portal genutzt, 
Teststrategien zur Funktionssicherung implementiert und eine modulare Softwarearchitektur entworfen. Abschließend wird die Dokumentation erstellt, die eine umfassende 
Beschreibung der Implementierung und Nutzung des Moduls sowie Anwendungsbeispiele und Testberichte enthält. Die Ergebnisse werden in einer Abschlusspräsentation vorgestellt. 
%//////////////////////////////////////////////////////////////////////////////////////////////////////////////////////
%//////////////////////////////////////////////////////////////////////////////////////////////////////////////////////
\section{Erwarteter Nutzen und Anwendungsbereich}
Das entwickelte Applikationsbeispiel bietet eine zukunftssichere Alternative zur Ablösung des FM350-2 Moduls und ermöglicht eine flexible Anpassung an industrielle Anforderungen. 
Ein konkretes Anwendungsbeispiel ist die Nachrüstung bestehender Anlagen, um eine Umstellung auf neue Hardware und Software zu erleichtern. Dies könnte insbesondere für Hersteller 
von Lebensmitteln oder pharmazeutischen Produkten von Interesse sein, die eine präzise Dosierung von Zutaten wie Aromen, Farbstoffen oder Vitaminen erfordern. 